In this essay I give a short overview of inductive reasoning in science.
I explain Karl Popper's view on the problems with induction,
and present a handful of objections to his solution, which is
\emph{falsificationism}.
Finally, I explore how Thomas Kuhn's \emph{paradigm shifts}, which
give a better account how science historically worked,
can work together with Popper's ideas, if
we apply both concepts more pragmatically.

It is not a bold claim to say that most sciences
follow
an empiricist methodology:
Scientists test their theories and hypothesis by doing experiments.
But how do we generally justify
any conclusion?
According to David Hume, there are only two ways of doing so.
The first one is by using a deductive argument.
A deductive argument concludes the outcome from a
more general statement. Here, the truth of the premise
guarantees the truth of the conclusion, it is truth-preserving.
Inductive inferences on the other hand, proceed from
the specific to the more general and are not
truth-preserving.
One famous example is the swan-induction:
If I observe a hundred white swans and no black ones, I may conclude
that all swans are white.
This reasoning is obviously not deductively valid, since
there is no logical reason to why the 101st swan can not be black.
Nonetheless, inductive reasoning seems to work in a lot of cases, as
science builds its foundation on empirical evidence.

But if inductive arguments are not logically valid, what makes them work?
Hume argues there are two different possibilities to vindicate induction.
We can either employ an inductive argument or a deductive argument.
The former fails, since induction is
the very thing we want to conclude, we have no grounds on
using it in the first place.
The latter means we need a premise from which to conclude the
validity of induction.
From a premise like \enquote{Things which have been proven to hold true in
    the past, will continue to do so in the future.} we can derive
that induction is valid, since it has proven to hold in the past.
But the premises of this deductive argument
seems to be justified in our experience that
\enquote{The future will be similar to the past}, which
is only inductively valid.
This is obviously question-begging and what
David Hume identified as the riddle of induction\cite[173]{philsciencebook}.

Sir Karl Popper took Hume's riddle of induction very seriously.
With \emph{falsificationism}, he wants to avoid the
problems of induction by arguing they are only pseudo-problems,
since science does not really employ inductive reasoning after all.
Popper probably still remains today the most influential philosopher
among scientists.
He was a close friend to the Vienna circle of the logical empiricist,
although not a constant member.
As they found themselves in the rise of European fascism, which
legitimized its views also with pseudo-science,
Popper developed a demarcation criterion to distinguish
between science and pseudo-science: falsification.
According to Popper, what makes a hypothesis scientific is \emph{not}
the recurring employment of inductive reasoning, by
finding more and more instances which confirm a theory,
since Hume's riddle of induction shows that it
is not logically valid.
Instead, a scientific theory is \emph{falsifiable}, meaning
it predicts at least one risky consequence which
can theoretically turn out to be wrong.
Science is then the process of constantly testing
hypotheses by trying to disproof them.
The hypothesis which survives the extensive
attempts to proof it wrong is a good scientific hypothesis.

At this point one might wonder:
\enquote{But does falsificationism hold up to its own standards and is falsifiable?}.
It is not, but according to Popper this is not a problem.
Falsifiability only distinguishes between science and not-science, not
between right or wrong.
Opposite to the logical empiricist he did not reject everything
unscientific as mere unuseful fiction:
He held that metaphysical programs had their validity, but not in science.

How does falsification solve the problems with induction?
It does not.
Instead, it takes them seriously and
employs a deductive methodology for science instead, since
falsification is deductively valid:
If we observe a black swan, it is deductively valid to conclude
that the hypothesis \enquote{All swans are white.} is wrong --
by using the power of counter-examples.
From this perspective science does not
give evidence that a hypothesis is true,
instead the best it can do is to claim that a hypothesis is not
falsified yet, and has withstood severe testing.
Take general relativity (GR) as an example for this principle.
From the very beginning Einstein provided a demarcation criterion for his theory:
If one is not to observe the bending of light by gravitational fields, his theory
would be clearly wrong.
Otherwise, it is at least able to explain new phenomena,
which Newtons theory could not account for, thus Newtonian gravity was falsified.
This was only the beginning of the tests GR
has withstood. Since then, a lot of bold predictions by GR were confirmed
(such as the existence of gravitational waves, black holes and so on).
In the case they would have not been observed, GR would be questioned by physicists, and
ultimately replaced -- that is, if
science strictly works by Popper's ideals.

But does science really work like this?
As a description of how science actually operates Popper's
philosophy seems to fail.
For example, when scientists found in the 19th century that Jupiter and Saturn
did not follow Newtons law of gravitation, we did not overthrow Newtonian gravity.
We blamed the \emph{auxiliary assumption} that there were no other planets we know of --
and thus Neptune and Uranus were postulated -- and eventually found \cite[193]{philsciencebook}.
Even nowadays with GR one can find deviation from
strict falsificationism.
We observe that the outer mass of galaxies does not
follow GR predictions, it moves to fast.
Instead of overthrowing GR and searching for a new theory,
the leading explanation in the scientific community
blames the auxiliary assumption, that there is no other mass and
proposes
\emph{dark matter}\footnote{There is other evidence supporting the dark-matter thesis, I simplified a bit.}.
The alternative of introducing a new theory
-- a possible candidate in the community is modified newton dynamics (MOND) --
is less favored among physicists.

This also shows the \emph{underdetermination} of falsification:
if we make an observation in conflict with our hypothesis, we most often don't
know what to reject exactly: usually there are a lot of auxiliary hypotheses
or parameters one can choose to reject.
This leads us to expect that a falsified scientific theory,
splitters into multiple new theories, each adjusting for new evidence with
different methods. Some will adopt completely new methods, some will
adjust auxiliaries.
But this does not seem to describe the state of science we observe:
there are rarely equivalently favored descriptive hypotheses at the same time
for a given phenomenon.

Thomas Kuhn set out to provide a description of how scientific
research actually works in his 1962 Book \enquote{The structure of scientific revolutions}.
Roughly speaking, he says that a science starts out as a proto-science, and
after adopting an excepted \emph{paradigm} among researchers,
it becomes a science.
Then, contrary to Popper, science does not progress in a linear
fashion by applying a well-formulated scientific method (like falsification),
but undergoes cycles of \emph{paradigm-shifts} where the central paradigm
is replaced, after too many anomalies build up.
With this Kuhn criticizes the descriptive implications
of Poppers falsificationism.

Kuhn's \emph{paradigm shifts} indeed provide a much better description of
how science actually operates. Kuhn recognizes that
Popper never intended to provide a description of scientific history,
but searched a normative guideline for good science.
Nonetheless, this has descriptive implications.
A prominent example of a scientific theory, which is
generally held as one of the most excellent scientific advancements,
but fails to be falsifiable is Darwin's theory of evolution.
Popper himself recognized it, and gave the explanation that
\enquote{Origin of Species} has to be seen not directly as science, but
as a metaphysical research program\cite[197]{philsciencebook}. This does not seem to be satisfactory.
But I argue, that if we both take Kuhn's description of paradigm shifts
and Poppers falsificationism in a less extreme way, they provide
a good guideline to understand the descriptive power of science and
to distinguish good from bad science.

I argue, that
falsification is not the way science arrives at a hypothesis or theory.
It is more applicable as the ultimate test for theories,
in a fully developed paradigm, once we believe
that our theories should adequately describe nature.
I think this can nicely align with Kuhn's paradigm shifts:
A paradigm shift can happen when the leading paradigm
is falsified.

I also think that in a lot of cases, falsification actually
provides a great description of science.
Going back to the dark-matter example:
Since some matter does not follow the laws of GR,
scientists developed the auxiliary assumption of dark-matter, but
they also developed MOND.
The fact that completely overthrowing GR with MOND is
not the most favored option in the community right now only reflects
that the paradigm is not completely developed.
The astronomy community is actively working on
falsifying MOND and if MOND survives further testing, it is completely
thinkable that it becomes the new accepted theory of gravitation.
There are also a lot of instances where theories fell in
disfavor in the scientific community, because they were falsified.
For example: Before the large hadron collider (LHC) in CERN began operating,
super symmetry (SUSY) was a beloved theory among most particle-theorists, which
predicted the existence of SUSY-particles.
It was believed that the LHC will find these particles -- but it did not.
As a result the SUSY theory is now generally disfavored in the community.

To summarize, this essay gave a short explanation about induction and its problems.
I continue by explaining how Popper tackles these problems, and then by
providing criticism of Popper through the descriptive
views on science developed by Kuhn.
In the end I started to explore how these two models can work together,
if they are interpreted in a more pragmatic way.