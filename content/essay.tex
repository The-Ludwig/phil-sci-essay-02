Here comes an abstract/introduction -----


It is not a bold claim to say that most sciences,
or what is generally excepted to be science, follow
an empiricist methodology:
Scientists test their theories and hypothesis by doing experiments.
Even rationalists, although rejecting that \emph{all}
knowledge is justified by experience,
they still believed \emph{some} knowledge is gained through experience
\cite[9]{philsciencebook}.

How do we test a theory, or how do we generally justify
any conclusion?
According to David Hume, there are only two ways of doing so.
The first one is by using a deductive argument.
A deductive argument concludes the outcome from a
more general statement, and the truth of the general statement
guarantees the truth of the conclusion. Thus, the reasoning
is truth-preserving.
[[Maybe an example?]]
Inductive inferences on the other hand, proceed from
the specific example to the more general and are not
truth-preserving.
One famous example is the swan-induction:
If I observe a hundred white swans and no black ones, I may conclude
that all swans are white.
This reasoning is obviously not deductively valid, since
there is no logical reason to why the 101th swan can't be black.
Nonetheless, inductive reasoning seems to
work great.
Science builds its foundation on empirical evidence for its
claims and thus using inductive arguments and even in everyday
life inductive arguments are a powerful guide.
But if inductive arguments are not logically valid, what makes them work?

Hume argues there are two different possibilities to vindicate induction.
We can either employ an inductive argument, but since that is
the very thing we want to conclude, we have no grounds on
using it in the first place, or we can use a deductive argument.
This means we need a general hypothesis from which to conclude the
validity of induction.
From something like \enquote{Things which have been proven to hold true in
    the past, will continue to do so in the future.} we can derive
that induction is valid, since it has proven to hold true in the past,
and with this law we can derive it will continue to do so in the future.
But the premises of the deductive argument to vindicate induction
seem to include statements we think of as true because of inductive
reasoning. This is obviously question-begging.
This is David Hume's riddle of induction.
