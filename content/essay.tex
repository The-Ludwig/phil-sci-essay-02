Here comes an abstract/introduction -----

\section{Induction and Empiricism}

It is not a bold claim to say that most sciences,
or what is generally excepted to be science, follow
an empiricist methodology:
Scientists test their theories and hypothesis by doing experiments.
Even rationalists, although rejecting that \emph{all}
knowledge is justified by experience,
they still believed \emph{some} knowledge is gained through experience
\cite[9]{philsciencebook}.

How do we test a theory, or how do we generally justify
any conclusion?
According to David Hume, there are only two ways of doing so.
The first one is by using a deductive argument.
A deductive argument concludes the outcome from a
more general statement, and the truth of the general statement
guarantees the truth of the conclusion. Thus, the reasoning
is truth-preserving.
[[Maybe an example?]]
Inductive inferences on the other hand, proceed from
the specific example to the more general and are not
truth-preserving.
One famous example is the swan-induction:
If I observe a hundred white swans and no black ones, I may conclude
that all swans are white.
This reasoning is obviously not deductively valid, since
there is no logical reason to why the 101th swan can't be black.
Nonetheless, inductive reasoning seems to
work great.
Science builds its foundation on empirical evidence for its
claims and thus using inductive arguments and even in everyday
life inductive arguments are a powerful guide.

A proposed formalization of a scientific method is the
hypothetico-deductive (HD) model. Roughly speaking,
in the HD model science starts by formulating a research question,
which addresses some observable phenomenon.
Science then formulates a hypothesis $H$, so that if the hypothesis is true,
it can explain the observed phenomenon. From $H$, scientists
derive another testable prediction and compare that prediction
to empirical observation.
If the observation is in contradiction with the hypothesis, we reject it,
and if it agrees, we have \emph{partially} confirmed it.
Notice the emphasis on \enquote{partially} there: due to the
non-truth preserving nature of induction, we can not confirm the
hypothesis with the HD method.


\section{Proofing induction}

But if inductive arguments are not logically valid, what makes them work?
Hume argues there are two different possibilities to vindicate induction.
We can either employ an inductive argument, but since that is
the very thing we want to conclude, we have no grounds on
using it in the first place, or we can use a deductive argument.
This means we need a premise from which to conclude the
validity of induction.
From a premise like \enquote{Things which have been proven to hold true in
    the past, will continue to do so in the future.} we can derive
that induction is valid, since it has proven to hold in the past.
% and with this law we can derive it will continue to do so in the future.
But the premises of this deductive argument
seems to be justified in our experience that
\enquote{The future will be similar to the past}, which is
valid by induction.
This is obviously question-begging and what
David Hume identified as a riddle of induction.
Any obvious further attempts to justify induction
ultimately end up in circular reasoning by applying inductive arguments to
vindicate induction.
Hume did not think of this argument as the end of induction, merely
he believed, that the right reasoning to vindicate induction
still has to be found \cite[173]{philsciencebook}.

% this might be ooptional
Sadly, this is not the only problem with the HD model.
There arises an interesting problem if we ask what kind of empirical evidence
should count toward partially confirming a hypothesis.
Think of our black swan example again. Let us logically rephrase
the hypothesis \enquote{All swans are white.} (H1)
to \enquote{All non-white things are not swans.} (H2).
% puhhh, i made better arguments before
To see that H1 and H2 are logically equivalent statements, think of a counter example:
if we see a non-white thing, which is a swan, we can reject the H1, but
we also encountered a swan which is not white, so we can also reject H2.
Now, if we \enquote{measure} a blue-yellow flag, we have encountered a
non-white thing, which is not a swan, so clearly evidence for H2.
But since H1 and H2 are equivalent, we also partially confirmed H1.
Can the existence of blue-yellow flags confirm \enquote{All swans are white.}?
This thought is at least very counter-intuitive and this \enquote{
    paradox of confirmation
} was created by Carl G. Hempel \cite[193]{philsciencebook}.
Hempel suggested that the right way to move on is to simply
accept, that a blue-yellow flag counts as a positive instance of
\enquote{All swans are white.}.

The problems of induction do not end here, even if we believe that
the paradox of confirmation is just an unimportant hypothetical
construct.
    [[[maybe explain the grue emerald here]]]

\section{Falsificationism}
Sir Karl Popper took Hume's riddle of induction very seriously.
With his \emph{falsificatonism} he sought out to eliminate the
problems with induction by arguing they are pseudo-problems after all.
Popper probably still remains today the most influential philosopher of science
among scientists.
To understand Popper's philosophy and his solution to the problems of induction,
it is good to understand where he was coming from.
He was a close friend to the Vienna circle of the logical empiricist,
although not a constant member.
As they found themselves in the rise of European fascism, which
legitimized their views with a lot of pseudo-science,
Popper developed a demarcation criterion to distinguish
between science and pseudo-science: falsification.
According to Popper, what makes a Hypothesis scientific is \emph{not}
the recurring employment of inductive reasoning, by
finding more and more instances which confirm a theory,
since Hume's riddle of induction shows that it
is not logically valid.
Instead, a scientific theory is \emph{falsifiable},
it predicts at least one risky consequence which
can theoretically turn out to be wrong.
Science is then the process of constantly testing and
questioning hypotheses by trying to dis-proof them
-- the hypothesis which survives the extensive
attempt to proof it wrong is a good scientific hypothesis.

At this point one might wonder:
\enquote{But does falsificationism hold up to its own standards and is falsifiable?}.
It is not, but according to Popper this is not a problem.
Falsifiability only distinguishes between science and not-science, not
between right or wrong.
Opposite to the logical empiricist he did not reject everything
unscientific as mere unuseful fiction:
He held that metaphysical programs had their validity, but not in science.

How does this solve the problems with induction?
It doesn't really -- it just takes them seriously and
employs a deductive methodology for science instead:
If we observe a black swan, it is deductively valid to conclude
that the hypothesis \enquote{All swans are white.} is wrong --
by using the power of counter-examples.
From this perspective science does not
give evidence that a hypothesis is true,
instead the best it can do is to claim that a hypothesis is not
wrong (yet), and has withstood severe testing.
Let's see how this unmasks for example homeopathy as a
pseudo-science, compared to for example the general theory of relativity.
Homeopathy is a pseudo-medicine which operates by giving very diluted
versions of substances which cause the symptoms it tries to cure; if
you have fever you are treated by consuming a very diluted
fever-inducing plant, like toadstool.
The dilution is so high, that it is very likely that in a given treatment,
the patient might not consume any molecule of the plant at all.
The conclusion that any apparent effects of this treatment are due to
mere believe and thus the placebo effect is natural.
But many homeopathists argue, that one is not to disproof homeopathy
with a double-blinded study to rule out any placebo-effects\footnote{
    All double-blinded studies that were reasonably good in quality showed no effect of
    homeopathy compared to placebo medicin\cite{mathie2017randomised}.},
since the patients \enquote{spirit} plays and important role in the treatment,
and thus the patient needs to know if he or she received the homeopathy or the placebo.
But this leaves no room to falsify homeopathy, and homeopathists do not provide
a situation in which they are willing to give up homeopathy and alter their treatment --
it is not falsifiable.
On the other side of the spectrum there is the general theory of relativity (GR),
one of the most successful theories in physics and one that Popper himself
held as one the most excellent applications of the scientific method as he
described it -- as a matter of fact, it inspired him to come up with falsifications
in the first place.
From the very beginning Einstein provided a demarcation criterion for his theory:
If one is not to observe the bending of light by gravitational fields, his theory
would be clearly wrong. And this was only the beginning of the falseifiaction tests GR
has withstood. Since then a lot of bold predictions by GR were confirmed
(such as the existence of gravitational waves, black holes and so on.).
In the case they would have not been confirmed, GR would be questioned by physicists, and
ultimately replaced.

\subsection{Objection to falsification}
But does science really work like this? Will we really overthrow GR
in the case we find it falsified by evidence?
As a description of how science actually operates Poppers
philosophy fails rigorously.
For example, when we found with better observations in the 19th century that Jupiter and Saturn
did not follow Newtons law of gravitation, we did not overthrow Newtonian gravity.
We blamed the \emph{auxiliary} that there were no other planets we know of --
so Neptune and Uranus were found.
Even nowadays with general relativity you can find deviation from
radical falsificationism.
We observe that the outer mass of galaxies does not
follow GR predictions, it moves to fast.
Instead of overthrowing GR and searching for a new theory,
the leading explanation in the scientific community
blames the auxiliary assumption, that there is no mass we can't see, so
we propose
\emph{dark matter}\footnote{There is other evidence supporting the dark-matter thesis, I simplified a bit.}.
The alternative of introducing a new theory
-- a possible candidate in the community is modified newton dynamics (MOND) --
is less favored among physicists.
This shows the \emph{underdetermination} of falsification:
if we make an observation in conflict with our hypothesis, we most often don't
know what to reject exactly: usually there are a lot of auxiliary hypotheses
or parameters one can choose to reject.
This leads us to expect that a scientific theory which becomes falsified,
splitters into multiple new theories, each adjusting for new evidence with
different methods: some will adopt completely new methods, some will
adjust auxiliaries.
But this does not seem to describe the state of science we observe:
there are rarely equivalently descriptive hypotheses at the same time
for a given phenomenon.

Thomas Kuhn set out to provide a description of how scientific
research actually works in his 1962 Book \enquote{The structure of scientific revolutions}.
Roughly speaking, he says that a science starts out as a proto-science, and
after adopting an excepted \emph{paradigm} among researchers,
it becomes a science.
Then, contrary to Popper, science does not progress in a linear
fashion by applying a well-formulated scientific method (like falsification),
but undergoes cycles of \emph{paradigm-shifts} where the central paradigm
is replaced, after too many anomalies build up.
With this Kuhn criticizes the descriptive implications
of Poppers falsificationism.

\subsection{In defense of falsification}
Kuhn's \emph{paradigm shifts} indeed provide a much better description of
how science operates than Popper. Kuhn regonizes that
Popper never intended to provide a description of scientific history,
but searched a normative explanation, of what is good science.
And even there a lot of what is generally believed to be good science is ...